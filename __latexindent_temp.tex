%!TEX program = xelatex
%表示用xelatex编译文件
\documentclass[a4paper]{ctexart}
\usepackage{array}
\usepackage{tabularx}
\usepackage{indentfirst} 
\setlength{\parindent}{2em}
\usepackage{booktabs}
\usepackage{multirow}
\usepackage{makecell}
\usepackage{graphicx}
\usepackage{subfig}
\usepackage{amsmath}
\usepackage{amsfonts}
\usepackage{amssymb}
\usepackage{amsthm}
\usepackage{latexsym}
\begin{document}
    \title{标题页}
    \author{Ryan\thanks{注脚}%
        \and Fan\thanks{注脚}%
        }
    \date{\today}
    \maketitle
    \abstract
    一般用于紧跟\textbackslash maketitle 命令之后介绍文档的摘要\par
    中文\LaTeX{}排版。
    \section{用\LaTeX 排版文字}
    {}分段\\换行\textbackslash\textbackslash\par
    冒号``please press the `x' key.''\par
    连字符-用来组成复合词,\\%
    短破折号--用来连接数字表示范围,\\%
    长破折号---用来连接单词\par
    省略号\dots{}和\ldots\par
    波浪号~\par
    强调\underline{文字,但是无法换行,}%
    \uline{ulem 宏包解决了这一问题,它提供的 uline 命令能够轻松生成自动换行的下划线。}%
    \emph{emph 命令用来将文字变为斜体以示强调。%
        \emph{如果在本身已经用 emph 命令强调的文字内部嵌套使用 emph 命令,}%
        内部则使用直立体文字。%
        }\par
    在合适的位置插入一个不会断行的空格Fig.~1, Ryan~Fan\par
    断行\\[15pt]可以带可选参数 $\langle length\rangle$,用于在换行处向下增加垂直间距%
    \newline{}或者newline命令,不用带参数\par
    \newpage 断页,在在双栏排版中只起到另起一栏的作用\par
    断词 I think this is: supercalifragil\-isticexpialidocious. %
    And I think this is: supercalifragilisticexpialidocious.\par
    \newpage
    \tableofcontents
    \section[目录和页眉页脚]{章节}
    \subsection{子章节}
    \subsubsection{子子章节}
    \paragraph{段落}
    \subparagraph{子段落}
    \section*{标题不带编号}
    \addcontentsline{toc}{section}{标题不带编号}
    \part{分块}
    \section{交叉引用}
    A reference to this subsection\label{sec:this} looks like: %
    ``see section~\ref{sec:this} on page~\pageref{sec:this}.''
    \section{脚注和边注}
    “天地玄黄,宇宙洪荒。日月盈昃,辰宿列张。”\footnote{出自《千字文》。}\par
    \begin{tabular}{l}
        \hline
        有些情况下(比如在表格环境、各种盒子\\
        内)使用 footnote 并不能正确生成脚\\
        注。我们以分两步进行,先使用 \\
        footnotemark 为脚注计数,再在合适\\
        的位置用 footnotetext 生成脚注。\footnotemark\\
        \hline
    \end{tabular}
    \footnotetext{出自《千字文》。}
    \marginpar{\footnotesize 边注较窄,不要写过多文字,最好设置较小的字号。}
    \section{特殊环境}
    \subsection{列表}
    有序列表
    \begin{enumerate}
        \item An item.
        \begin{enumerate}
            \item A nested item.\label{itref}
            \item[*] A starred item.
        \end{enumerate}
        \item Reference(\ref{itref}).
    \end{enumerate}
    无序列表
    \begin{itemize}
        \item An item.
        \begin{itemize}
            \item A nested item.
            \item[+] A `plus' item. + A ‘plus’ item.
            \item Another item. – Another item.
        \end{itemize}
        \item Go back to upper level.
    \end{itemize}
    关键字环境
    \begin{description}
        \item[Enumerate] Numbered list.
        \item[Itemize] Non-numbered list.
    \end{description}
    重定义无序列表的符号
    \renewcommand{\labelitemi}{\dag}
    \renewcommand{\labelitemii}{\ddag}
    \begin{itemize}
        \item First item
        \begin{itemize} 
            \item Subitem
            \item Subitem
        \end{itemize}
        \item Second item
    \end{itemize}
    重定义有序列表的符号
    \renewcommand{\labelenumi}{\Alph{enumi}>}
    \begin{enumerate}
        \item First item
        \item Second item
    \end{enumerate}
    \subsection{对齐环境}
    center、flushleft 和 flushright 环境分别用于生成%
    居中、左对齐和右对齐的文本环境。
    \begin{center}
        Centered text using a
        \verb|center| environment.
    \end{center}
    \begin{flushleft}
        Left-aligned text using a
        \verb|flushleft| environment.
    \end{flushleft}
    \begin{flushright}
        Right-aligned text using a
        \verb|flushright| environment.
    \end{flushright}
    还可以用以下命令直接改变文字的对齐方式:
    \centering
    Centered text paragraph.\par
    \raggedright
    Left-aligned text paragraph.\par
    \raggedleft
    Right-aligned text paragraph.\par
    \begin{flushleft}
        center 等环境会在上下文产生一个额外间距,%
        而 \textbackslash centering 等命令不产生,只是改变对齐方式。%
    \end{flushleft}
    \raggedright
    比如在浮动体环境 table 或 figure 内实现居中对齐,%
    用 \textbackslash centering 命令即可,没必要再用 center 环境。
    \subsection{引用环境}
    \begin{description}
        \item[quote] 用于引用较短的文字,首行不缩进\\
        Francis Bacon says:
        \begin{quote}
            Knowledge is power.
        \end{quote}
        \item[quotation] 用于引用若干段文字,首行缩进\\
        《木兰诗》:
        \begin{quotation}
            万里赴戎机,关山度若飞。
            朔气传金柝,寒光照铁衣。
            将军百战死,壮士十年归。

            归来见天子,天子坐明堂。
            策勋十二转,赏赐百千强。......
        \end{quotation} 
    \end{description}
    \subsection{代码环境}
    \begin{verbatim}
        #include <iostream>
        int main() 
        {
        std::cout << "Hello, world!"
                    << std::endl;
        return 0;
        }
    \end{verbatim}
    \begin{verbatim*}
        for (int i=0; i<4; ++i)
        printf("Number %d\n",i);
    \end{verbatim*}
    要排版简短的代码或关键字\textbackslash verb $\langle delim\rangle\langle code\rangle\langle delim\rangle$\par
    $\langle delim\rangle$ 标明代码的分界位置,前后必须一致,除字母、空格或星号外,%
    可任意选择使得不与代码本身冲突,习惯上使用 | 符号。\par
    \verb|\LaTeX| \\ 
    \verb+(a || b)+ \verb*+(a || b)+    
    \subsection{表格}
    \subsubsection{列表格}
    tabular 环境使用 $\langle column-spec\rangle$ 参数指定表格的列数以及每列的格式。\par
    \begin{tabular}{lcr|p{6em}}
        \hline
        left & center & right & par box with fixed width\\
        L    & C      & R     & P\\
        \hline
    \end{tabular}\par
    @ 格式可在单元格前后插入任意的文本,%
    但同时它也消除了单元格前后额外添加的间距。\par
    \begin{tabular}{@{} r@{:}lr @{}}
        \hline
        1 & 1 & one\\
        11 & 3 & eleven\\
        \hline
    \end{tabular}\par
    格式参数重复\par
    \begin{tabular}{|*{5}{c|}*{2}{p{3em}|}}
        \hline
        one & two & three & four & five & Hello! \LaTeX & Hello!\\
        1   & 2   & 3     & 4    & 5    & hello!        & \LaTeX\\
        \hline    
    \end{tabular}\par
    辅助格式 > 和 <,用于给列格式前后加上修饰命令\par
    \begin{tabular}{>{\itshape}r<{*}l}
        %需要使用array宏包
        \hline
        italic & normal \\
        column & column \\
        \hline    
    \end{tabular}\par
    \begin{tabular}{>{\centering\arraybackslash}p{16em}}
        \hline
        辅助格式甚至支持插入 \textbackslash centering 等%
        命令改变 p 列格式的对齐方式,一般还要加额外的命令 %
        \textbackslash arraybackslash 以免出错。\\
        \hline
        \textbackslash centering 等对齐命令会破坏表格环境里 %
        \textbackslash\textbackslash 换行命令的定义,%
        \textbackslash arraybackslash 用来恢复之。%
        如果不加 \textbackslash arraybackslash 命令,%
        也可以用 \textbackslash tabularnewline 命令%
        代替原来的 \textbackslash\textbackslash 实现表格换行。\\
        \hline
    \end{tabular}\par
    \LaTeX 本身提供了 tabular* 环境用来排版定宽表格,但是不太方便使用,%
    比如要用到 @ 格式插入额外命令,令单元格之间的间距为 \textbackslash fill,%
    但即使这样仍然有瑕疵:\par
    \begin{tabular*}{14em}{@{\extracolsep{\fill}}|c|c|c|c|}
        \hline
        A & B & C & D \\
        \hline
        a & b & c & d \\
        \hline
    \end{tabular*}\par
    tabularx 宏包为我们提供了方便的解决方案。它引入了一个 X 列格式,%
    类似 p 列格式,不过会根据表格宽度自动计算列宽,多个X列格式平均分配列宽。%
    X列格式也可以用 array 里的辅助 格式修饰对齐方式:\par
    \begin{tabularx}{14em}{|*{4}{>{\centering\arraybackslash}X|}}
        \hline
        A & B & C & D \\ 
        \hline 
        a & b & c & d \\ 
        \hline
    \end{tabularx}\par
    \subsubsection{横线}
    \textbackslash cline\{$\langle i-j\rangle$\} 用来绘制跨越部分单元格的横线:\par
    \begin{tabular}{|c|c|c|}
        \hline
        4 & 9 & 2 \\ \cline{2-3}
        3 & 5 & 7 \\ \cline{1-1}
        8 & 1 & 6 \\ 
        \hline
    \end{tabular}\par
    三线表由 booktabs 宏包 支持,它提供了 \textbackslash toprule、%
    \textbackslash midrule 和 \textbackslash bottomrule %
    命令用以排版三线表的三条线, 以及和 \textbackslash cline 对应的 %
    \textbackslash cmidrule。除此之外,最好不要用其它横线以及竖线:\par
    \begin{tabular}{cccc}
        \toprule
        & \multicolumn{3}{c}{Numbers} \\
        \cmidrule{2-4}
                & 1 & 2 & 3 \\
        \midrule
        Alphbet & A & B & C \\
        Roman   & I & II& III \\
        \bottomrule
    \end{tabular} \par
    \subsubsection{合并单元格}
    横向合并单元格较为容易,由 \textbackslash multicolumn\{$\langle n\rangle$\}\{$\langle column-spec\rangle$\}\{$\langle item\rangle$\} %
    命令实现:\par
    其中 $\langle n\rangle$ 为要合并的列数,$\langle column-spec\rangle$ 为合并单元格后的列格式,只允许出现一个 l/c/r 或 p 格式。%
    如果合并前的单元格前后带表格线 |,合并后的列格式也要带 | 以使得表格的竖线一致。\par
    形如 \textbackslash multicolumn\{1\}\{$\langle column-spec\rangle$\}\{$\langle item\rangle$\} %
    的命令可以用来修改某一个单元格的列格式。\par
    \begin{tabular}{|c|c|c|}
        \hline
        1 & 2 & Center \\
        \hline
        \multicolumn{2}{|c|}{3} & \multicolumn{1}{r|}{Right} \\
        \hline
        4 & \multicolumn{2}{c|}{C} \\
        \hline
    \end{tabular}\par
    纵向合并单元格需要用到 multirow 宏包提供的 %
    \textbackslash multirow\{$\langle n\rangle$\}\{$\langle width\rangle$\}\{$\langle item\rangle$\} 命令:\par
    $\langle width\rangle$ 为合并后单元格的宽度,可以填 * 以使用自然宽度。\par
    \begin{tabular}{ccc}
        \hline
        \multirow{2}{*}{Item} & \multicolumn{2}{c}{Value} \\ 
        \cline{2-3}
                              & First & Second \\
        \hline
                            A & 1     & 2 \\
        \hline 
    \end{tabular}
    \subsubsection{嵌套表格}
    在单元格中嵌套一个小表格可 以起到``拆分单元格''的效果。\par
    注意要用 \textbackslash multicolumn 命令配合 @\{\} %
    格式把单元格的额外边距去掉,使得嵌套的表格线能和外层的表格线正确相连:\par
    \begin{tabular}{|c|c|c|}
        \hline
        a & b                                   & c \\
        \hline
        a & \multicolumn{1}{@{}c@{}|}
                        {\begin{tabular}{c|c}
                            e & f \\
                            \hline
                            e & f \\
                         \end{tabular}}         & c \\
        \hline
        a & b                                   & c \\
        \hline
    \end{tabular}\par
    如果不需要为“拆分的单元格”画线,并且只在垂直方向“拆分”的话,makecell 宏包%
    提供 的 \textbackslash makecell 命令是一个简单的解决方案:\par
    \begin{tabular}{|c|c|}
        \hline
        a & \makecell{d1 \\ d2} \\
        \hline
        b & c \\
        \hline
    \end{tabular}
    \subsubsection{行距控制}
    \LaTeX 生成的表格看起来通常比较紧凑。%
    修改参数 \textbackslash arraystretch 可以得到行距更加宽松 的表格:\par
    \renewcommand\arraystretch{1.8}
    \begin{tabular}{|c|}
        \hline
        Really loose \\
        \hline
        tabular rows. \\
        \hline
    \end{tabular}\par
    另一种增加间距的办法是给换行命令 \textbackslash\textbackslash 添加可选参数,%
    在这一行下面加额外的间距,适合用于在行间不加横线的表格:\par
    \renewcommand\arraystretch{1}
    \begin{tabular}{c}
        \hline
        Head lines \\[6pt]
        tabular lines \\
        tabular lines \\
        \hline
    \end{tabular}
    \subsection{图片}
    \LaTeX 本身不支持插图功能,需要由 graphicx 宏包辅助支持。\par
    使用 \textbackslash includegraphics[$\langle options\rangle$]\{$\langle filename\rangle$\} 命令加载图片了:\par
    其中 $\langle filename\rangle$ 为图片文件名,文件名有时需要使用相对路径或绝对路径。\par
    \textbackslash graphicspath 命令,用于声明一个或多个图片文件存放的目录,%
    使用这些目录里的图片时可不用写路径:\par
    \textbackslash includegraphics 命令的可选参数 $\langle options\rangle$ 支持 %
    $\langle key\rangle$=$\langle value\rangle$ 形式赋值,常用的参数如下:\par
    \begin{tabular}{ll}
        \hline
        参数                                 & 含义 \\
        \hline
        width=$\langle width\rangle$        & 将图片缩放到宽度为$\langle width\rangle$ \\
        height=$\langle height\rangle$      & 将图片缩放到高度为$\langle height\rangle$ \\
        scale=$\langle scale\rangle$        & 将图片相对于原尺寸缩放$\langle scale\rangle$倍 \\
        angle=$\langle angle\rangle$        & 令图片逆时针旋转$\langle angle\rangle$度 \\
        \hline
    \end{tabular}
    \includegraphics[scale=0.05]{/home/ryan/Pictures/wallpaper/heic2007a}
    \graphicspath{{/home/ryan/Pictures/wallpaper/}}
    \includegraphics[scale=0.05]{heic1501a}
    \subsection{盒子}
    \subsubsection{水平盒子}
    \textbackslash mbox\{\ldots \}\par
    \textbackslash makebox$[\langle width\rangle][\langle align\rangle]\{\ldots \}$\par
    \textbackslash mbox 生成一个基本的水平盒子,内容只有一行,不允许分段(除非嵌套其它盒子)
    \textbackslash makebox 更进一步,可以加上可选参数用于控制盒子的宽度 $\langle width\rangle$,%
    以及内容的对齐方式$\langle align\rangle$,可选居中 c(默认值) 、左对齐 l、右对齐 r 和分散对齐 s\par
    |\mbox{Test some words.}|\\
    |\makebox[10em]{Test some words.}|\\
    |\makebox[10em][l]{Test some words.}|\\
    |\makebox[10em][r]{Test some words.}|\\
    |\makebox[10em][s]{Test some words.}|\par
    \subsubsection{带框的水平盒子}
    \textbackslash fbox 和 \textbackslash framebox 让我们可以为水平盒子添加边框。\par
    \fbox{Test some words.}\\
    \framebox[10em][r]{Test some words.}\par
    可以通过 \textbackslash setlength 命令调节边框的宽度 \textbackslash fboxrule %
    和内边距 \textbackslash fboxsep:\par
    \framebox[10em][r]{Test box.}\\
    \setlength{\fboxrule}{1.6pt}
    \setlength{\fboxsep}{1em}
    \framebox[10em][r]{Test box.}\par
    \subsubsection{垂直盒子}
    排版一个文字可以换行的盒子:\par
    \textbackslash parbox$[\langle align\rangle][\langle height\rangle][\langle inner-align\rangle]\{\langle width\rangle\}\{\ldots \}$
    \textbackslash begin\{minipage\}$[\langle align\rangle][\langle height\rangle][\langle inner-align\rangle]\{\langle width\rangle\}$\\
    \ldots\\
    \textbackslash end\{minipage\}\par
    其中 $[\langle align\rangle]$ 为盒子和周围文字的对齐情况(类似 tabular 环境); %
    $\langle height\rangle$ 和 $\langle inner-align\rangle$设置盒子的高度和内容的对齐方式,%
    类似水平盒子 \textbackslash makebox 的设置,不过 $\langle inner-align\rangle$ 接受的%
    参数是顶部 t、底部 b、居中 c 和分散对齐 s。\par
    三字经:\parbox[t]{3em}{人之初 性本善 性相近 习相远}
    \quad
    千字文:
    \begin{minipage}[b][8ex][t]{4em}
        天地玄黄 宇宙洪荒
    \end{minipage}\par
    如果在 minipage 里使用 \textbackslash footnote 命令,生成的脚注会出现在盒子底部,编号是独立的, %
    并且使用小写字母编号。而在 \textbackslash parbox 里无法正常使用 \textbackslash footnote 命令,%
    只能在盒子里使用\textbackslash footnotemark,在盒子外使用\textbackslash footnotetext。\par
    \fbox{这是一个垂直盒子的测试。\footnotemark}
    \footnotetext{注脚来自fbox}
    \fbox{\begin{minipage}{15em}%
             这是一个垂直盒子的测试。
             \footnote{注脚来自minipage.}   
          \end{minipage}
    }\par
    \subsubsection{标尺盒子}
    \textbackslash rule $[\langle raise\rangle]\{\langle width\rangle\}\{\langle height\rangle\}$%
    命令用来画一个实心的矩形盒子,也可适当调整以用来画线(标尺):\par
    Black \rule{12pt}{4pt} box.\\
    Upper \rule[4pt]{6pt}{8pt} and lower \rule[-4pt]{6pt}{8pt} box.\\
    A \rule[-.4pt]{6pt}{.4pt} line.\par
    \subsection{浮动体}
    \LaTeX 预定义了两类浮动体环境 figure 和 table。习惯上 figure 里放图片,table 里放表格,但并没有严格限制,%
    可以在任何一个浮动体里放置文字、公式、表格、图片等等任意内容。\par
    \textbackslash begin\{table\}$[\langle placement\rangle]$\\
    \ldots\\
    \textbackslash end\{table\}\par
    $[\langle placement\rangle]$ 参数提供了一些符号用来表示浮动体允许排版的位置,%
    如 hbp 允许浮动体排版在当前位置、底部或者单独成页。table 和 figure 浮动体的默认设置为 tbp。\par
    双栏排版环境下,\LaTeX 提供了 table* 和 figure* 环境用来排版跨栏的浮动体。它们的用%
    法与 table 和 figure 一样,不同之处为双栏的 $[\langle placement\rangle]$ 参数只能用 tp 两个位置。\par
    \begin{tabular}{ll}
        \toprule
        参数     & 含义\\
        \midrule
        h       & 当前位置(代码所处的上下文)\\
        t       & 顶部\\
        b       & 底部\\ 
        p       & 单独成页\\ 
        !       & 在决定位置时忽视限制\\
        \bottomrule
    \end{tabular}\par
    \textbackslash clearpage 命令 会在另起一页之前,先将所有推迟处理的浮动体排版成页,%
    此时 htbp 等位置限制被完全忽略。\par
    float 宏包为浮动体提供了 H 位置参数,不与 htbp 及 ! 混用。使用 H 位置参数时,%
    会取消浮 动机制,将浮动体视为一般的盒子插入当前位置。\par
    \subsubsection{浮动体的标题}
    图表等浮动体提供了 \textbackslash caption\{\ldots\} 命令加标题,并且自动给浮动体编号:\par
    可以用带星号的命令 \textbackslash caption* 生成不带编号 的标题,%
    也可以使用带可选参数的形式 \textbackslash caption[\ldots]\{\ldots\},
    使得在目录里使用短标题。\textbackslash caption 命令之后还可以紧跟 %
    \textbackslash label 命令标记交叉引用。\par
    可通过修改 \textbackslash figurename 和 \textbackslash tablename %
    的内容来修改标题的前缀。标题样式的 定制功能由 caption 宏包提供.\par
    table 和 figure 两种浮动体分别有各自的生成目录的命令:\\
    \textbackslash listoftables\\
    \textbackslash listoffigures\par    
    \subsubsection{并排和子图表}
    \begin{figure}[htbp]
        \centering
        \includegraphics[scale=0.015]{heic1501a}
        \qquad
        \includegraphics[scale=0.015]{heic2007a}\\
        \includegraphics[scale=0.02]{opo0501a}
        \caption{图片标题}
        \label{}
    \end{figure}
    由于标题是横跨一行的,用 \textbackslash caption 命令为每个图片单独生成标题%
    就需要借助前文提到的\textbackslash parbox 或者 minipage 环境,将标题限制在盒子内。\par
    \begin{figure}[htbp]
        \centering
        \begin{minipage}[b][120pt][t]{0.45\linewidth}
            \centering
            \includegraphics[scale=0.025]{heic1501a}
            \caption{并排图1}
        \end{minipage}
        \qquad
        \begin{minipage}[b][120pt][t]{0.45\linewidth}
            \centering
            \includegraphics[scale=0.025]{heic2007a}
            \caption{并排图2}
        \end{minipage}
    \end{figure}     
    给每个图片定义小标题时,就要用到 subfig 宏包的功能    
    \begin{figure}[htbp]
        \centering
        \subfloat[]{%
            \begin{minipage}[b][100pt][t]{0.45\linewidth}
                \centering
                \includegraphics[scale=0.025]{heic2007a}
            \end{minipage}
        }
        \qquad
        \subfloat[]{%
            \begin{minipage}[b][100pt][t]{0.45\linewidth}
                \centering 
                \includegraphics[scale=0.025]{heic1501a}
            \end{minipage}
        }
        \caption{使用 subfig 宏包的 \textbackslash subfloat 命令排版子图。}
    \end{figure}
    \section{数学公式}
    amsmath 宏 包,对多行公式的排版提供了有力的支持,%
    amsfonts 宏包以及基于它的 amssymb 宏包提 供了丰富的数学符号,%
    amsthm 宏包扩展了 \LaTeX 定理证明格式。\par
    \subsection{公式排版基础}
    \subsubsection{行内和行间公式}
    行内公式由一对 \$ 符号包裹:\\
    The Pythagorean theorem is %
    $a^2 + b^2 = c^2$.\par
    行间公式在 \LaTeX 里由 equation 环境包裹,equation 环境为公式自动生成一 个编号,%
    这个编号可以用 \textbackslash label 和 \textbackslash ref 生成交叉引用。\par
    amsmath 的 \textbackslash eqref 命令甚至为引用 自动加上圆括号;%
    还可以用 \textbackslash tag 命令手动修改公式的编号,%
    或者用 \textbackslash notag 命令取消为公式编 号。\par
    The Pythagorean theorem is:
    \begin{equation}
        a^2 + b^2 = c^2 \label{pythagorean}
    \end{equation}
    Equation \eqref{pythagorean} is called `Gougu theorem' in Chinese.\par
    It's wrong to say
    \begin{equation}
        1 + 1 = 3 \tag{dumb}
    \end{equation}
    or
    \begin{equation}
        1 + 1 = 4 \notag
    \end{equation}\par
    直接使用不带编号的行间公式:
    \begin{equation*}
        a^2 + b^2 = c^2
    \end{equation*}
    For short:
    \[a^2 + b^2 = c^2\]
    Or if you like the long one:
    \begin{displaymath}
        a^2 + b^2 = c^2
    \end{displaymath}
    行内公式和行间公式的对比:\\
    In text:
    $\lim_{n \to \infty} \sum_{k=1}^n \frac{1}{k^2} = \frac{\pi^2}{6}$\\
    In display:
    \[\lim_{n \to \infty} \sum_{k=1}^n \frac{1}{k^2} = \frac{\pi^2}{6}\]
    \subsection{数学模式}
    \renewcommand{\labelenumi}{\arabic{enumi}.}
    \begin{enumerate}
        \item 数学模式中输入的空格被忽略。
        \item 不允许有空行(分段)。
        \item 所有的字母被当作数学公式中的变量处理,想在数学公式中输入正体的文本,%
              可以用\textbackslash mathrm 或者 amsmath 提供的 \textbackslash text 命令。
    \end{enumerate}
    $x^2 \geq 0  \qquad
     \text{for \textbf{all} } x \in \mathbb{R}$
    \subsection{数学符号}
    \subsubsection{一般符号}
    \centering
    文本/数学模式通用符号\\
    \begin{tabular}{clclclcl}
        \hline
        \{      & \textbackslash\{       & \}   & \textbackslash\}    & \$          & \textbackslash\$             & \%     & \textbackslash\%  \\
        \dag    & \textbackslash dag     & \S   & \textbackslash S    & \copyright  & \textbackslash copyright     & \dots  & \textbackslash dots   \\
        \ddag   & \textbackslash ddag    & \P   & \textbackslash P    & \pounds     & \textbackslash pounds    \\
        \hline
    \end{tabular}
    \newpage
    希腊字母\par
    \begin{tabular}{clclclcl}
        \toprule
        $\alpha$        & \textbackslash alpha      & $\theta$      & \textbackslash theta      &
        $o$             & o                         & $\upsilon$    & \textbackslash upsilon    \\
        $\beta$         & \textbackslash beta       & $\vartheta$   & \textbackslash vartheta   &
        $\pi$           & \textbackslash pi         & $\phi$        & \textbackslash phi        \\
        $\gamma$        & \textbackslash gamma      & $\iota$       & \textbackslash iota       &
        $\varpi$        & \textbackslash varpi      & $\varphi$     & \textbackslash varphi     \\
        $\delta$        & \textbackslash delta      & $\kappa$      & \textbackslash kappa      &
        $\rho$          & \textbackslash rho        & $\chi$        & \textbackslash chi        \\
        $\epsilon$      & \textbackslash epsilon    & $\lambda$     & \textbackslash lambda     &
        $\varrho$       & \textbackslash varrho     & $\psi$        & \textbackslash psi        \\
        $\varepsilon$   & \textbackslash varepsilon & $\mu$         & \textbackslash mu         & 
        $\sigma$        & \textbackslash sigma      & $\omega$      & \textbackslash omega      \\
        $\zeta$         & \textbackslash zeta       & $\nu$         & \textbackslash nu         &
        $\varsigma$     & \textbackslash varsigma                                               \\ 
        $\eta$          & \textbackslash eta        & $\xi$         & \textbackslash xi         & 
        $\tau$          & \textbackslash tau                                                    \\
        \midrule
        $\Gamma$        & \textbackslash Gamma      & $\Lambda$     & \textbackslash Lambda     &
        $\Sigma$        & \textbackslash Sigma      & $\Psi$        & \textbackslash Psi        \\
        $\Delta$        & \textbackslash Delta      & $\Xi$         & \textbackslash Xi         &
        $\Upsilon$      & \textbackslash Upsilon    & $\Omega$      & \textbackslash Omega      \\
        $\Theta$        & \textbackslash Theta      & $\Pi$         & \textbackslash Pi         & 
        $\Phi$          & \textbackslash Phi                                                    \\
        \midrule
        \multicolumn{4}{l}{以下命令依赖 amsmath 宏包}\\
        $\varGamma$     & \textbackslash varGamma   & $\varLambda$  & \textbackslash varLambda  & 
        $\varSigma$     & \textbackslash varSigma   & $\varPsi$     & \textbackslash varPsi     \\
        $\varDelta$     & \textbackslash varDelta   & $\varXi$      & \textbackslash varXi      &
        $\varUpsilon$   & \textbackslash varUpsilon & $\varOmega$   & \textbackslash varOmega   \\
        $\varTheta$     & \textbackslash varTheta   & $\varPi$      & \textbackslash varPi      &
        $\varPhi$       & \textbackslash varPhi                                                 \\
        \midrule
        \multicolumn{4}{l}{依赖 amssymb 宏包}\\
        $\digamma$      & \textbackslash digamma    & $\varkappa$   & \textbackslash varkappa   &
        $\beth$         & \textbackslash beth       & $\gimel$‬      & \textbackslash gimel      \\ 
        $\daleth$       & \textbackslash daleth                                                 \\
        \bottomrule
    \end{tabular}\par
    其它符号\par
    \begin{tabular}{clclclcl}
        \toprule
        $\dots$         & \textbackslash dots       & $\cdots$          & \textbackslash cdots                          &
        $\vdots$        & \textbackslash vdots      & $\ddots$          & \textbackslash ddots                          \\
        $\hbar$         & \textbackslash hbar       & $\imath$          & \textbackslash imath                          &
        $\jmath$        & \textbackslash jmath      & $\ell$            & \textbackslash ell                            \\
        $\Re$           & \textbackslash Re         & $\Im$             & \textbackslash Im                             &
        $\aleph$        & \textbackslash aleph      & $\wp$             & \textbackslash wp                             \\
        $\forall$       & \textbackslash forall     & $\exists$         & \textbackslash exists                         &  
        $\partial$      & \textbackslash partial    & $'$               & '                                             \\ 
        $\prime$        & \textbackslash prime      & $\emptyset$       & \textbackslash emptyset                       &
        $\infty$        & \textbackslash infty      & $\nabla$          & \textbackslash nabla                          \\ 
        $\triangle$     & \textbackslash triangle   & $\bot$            & \textbackslash bot                            &
        $\top$          & \textbackslash top        & $\angle$          & \textbackslash angle                          \\
        $\surd$         & \textbackslash surd       & $\diamondsuit$    & \textbackslash diamondsuit                    &
        $\heartsuit$    & \textbackslash heartsuit  & $\clubsuit$       & \textbackslash clubsuit                       \\ 
        $\spadesuit$    & \textbackslash spadesuit  & $\neg$            & \textbackslash neg or \textbackslash lnot     &
        $\flat$         & \textbackslash flat       & $\natural$        & \textbackslash natural                        \\
        $\sharp$        & \textbackslash sharp                                                                          \\
        \midrule
        \multicolumn{4}{l}{以下命令依赖 latexsym 宏包}\\
        $\mho$          & \textbackslash mho        & $\Box$            & \textbackslash Box                            &
        $\Diamond$      & \textbackslash Diamond                                                                        \\
        \bottomrule
    \end{tabular}
    \flushleft
    $a_1,a_2,\dots,a_n$\\
    $a_1 + a_2 + \cdots + a_n$\par
    \subsubsection{指数、上下标和导数}
    \textasciicircum 和 \_ 标明上下标。%
    注意上下标的内容(子公式)一般需要用花括号包裹,否 则上下标只对后面的一个符号起作用:\par
    $p^3_{ij} \qquad
     m_\mathrm{Kunth} \qquad
     \sum_{k=1}^3 k$\\
    $a^x +y \neq a^{x+y} \qquad
     e^{x^2} \neq {e^x}^2$ \par
    导数符号$'$ 是一类特殊的上标,可以适当连用表示多阶导数,也可以在其后连用上标:\par
    $f(x) = x^2 \quad
     f'(x) = 2x \quad
     f''^{2}(x) = 4$ \par
    \subsubsection{分式和根式}
    分式使用 \textbackslash frac\{分子\}\{分母\} 来书写。%
    分式的大小在行间公式中是正常大小,而在行内被 极度压缩。%
    amsmath 提供了方便的命令 \textbackslash dfrac 和 \textbackslash tfrac,%
    令用户能够在行内使用正常大小的 分式,或是反过来。\par
    In display style:
    \[
        3/8 \qquad
        \frac{3}{8} \qquad
        \tfrac{3}{8}
    \]
    In text style:
    $1\frac{1}{2}$~hours \qquad
    $1\dfrac{1}{2}$~hours\par
    一般的根式使用 \textbackslash sqrt\{\dots\};%
    表示 n 次方根时写成 \textbackslash sqrt[n]\{\dots\}。\par
    $\sqrt{x} \Leftrightarrow x^{1/2} \quad
     \sqrt[3]{2} \quad
     \sqrt{x^{2} + \sqrt{y}}$ \par
    特殊的分式形式,如二项式结构,由 amsmath 宏包的 \textbackslash binom 命令生成:\par
    Pascal's rule is
    \[
        \binom{n}{k} = \binom{n - 1}{k} + \binom{n - 1}{k}   
    \]
    \newpage
    \subsubsection{关系符}
    \centering    
    二元关系符\par
    \begin{table}
        \caption{二元关系符}
        \begin{tabular}{clclcl}
            \toprule
            $<$             & <                         & $>$           & >                                         & 
            $=$             & =                                                                                     \\
            $\equiv$        & \textbackslash equiv      & $\leq$        & \textbackslash leq or \textbackslash le   & 
        $\geq$          & \textbackslash geq or \textbackslash ge                                               \\
        $\ll$           & \textbackslash ll         & $\gg$         & \textbackslash gg                         & 
        $\doteq$        & \textbackslash doteq                                                                  \\
        $\prec$         & \textbackslash prec       & $\succ$       & \textbackslash succ                       &
        $\sim$          & \textbackslash sim                                                                    \\
        $\preceq$       & \textbackslash preceq     & $\succeq$     & \textbackslash succeq                     &
        $\simeq$        & \textbackslash simeq                                                                  \\
        $\subset$       & \textbackslash subset     & $\supset$     & \textbackslash supset                     &
        $\approx$       & \textbackslash approx                                                                 \\
        $\subseteq$     & \textbackslash subseteq   & $\supseteq$   & \textbackslash supseteq                   &
        $\cong$         & \textbackslash cong                                                                   \\
        $\sqsubseteq$   & \textbackslash sqsubseteq & $\sqsupseteq$ & \textbackslash sqsupseteq                 & 
        $\bowtie$       & \textbackslash bowtie                                                                 \\
        $\in$           & \textbackslash in         & $\ni$         & \textbackslash ni ,\textbackslash owns    &
        $\propto$       & \textbackslash propto                                                                 \\
        $\vdash$        & \textbackslash vdash      & $\dashv$      & \textbackslash dashv                      &
        $\models$       & \textbackslash models                                                                 \\
        $\mid$          & \textbackslash mid        & $\parallel$   & \textbackslash parallel                   &
        $\perp$         & \textbackslash perp                                                                   \\
        $\smile$        & \textbackslash smile      & $\frown$      & \textbackslash frown                      &
        $\asymp$        & \textbackslash asymp                                                                  \\
        $:$             & :                         & $\notin$      & \textbackslash notin                      &
        $\neq$          & \textbackslash neq or \textbackslash ne                                               \\
        \midrule
        \multicolumn{4}{l}{以下命令依赖 latexsym 宏包}\\
        $\sqsubset$     & \textbackslash sqsubset   & $\sqsupset$   & \textbackslash sqsupset                   & 
        $\Join$         & \textbackslash Join                                                                   \\
        \midrule 
        \multicolumn{4}{l}{以下命令依赖 amssymb 宏包}\\
        $\lessdot$              & \textbackslash lessdot            & $\gtrdot$         & \textbackslash gtrdot         & 
        $\doteqdot$             & \textbackslash doteqdot                                                               \\
        $\leqslant$             & \textbackslash leqslant           & $\geqslant$       & \textbackslash geqslant       &
        $\risingdotseq$         & \textbackslash risingdotseq                                                           \\
        $\eqslantless$          & \textbackslash eqslantless        & $\eqslantgtr$     & \textbackslash eqslantgtr     &
        $\fallingdotseq$        & \textbackslash fallingdotseq                                                          \\
        $\leqq$                 & \textbackslash leqq               & $\geqq$           & \textbackslash geqq           &
        $\eqcirc$               & \textbackslash eqcirc                                                                 \\
        $\circeq$               & \textbackslash circeq             & $\ggg$            & \textbackslash ggg            &
        $\lll$                  & \textbackslash lll or \textbackslash llless                                           \\
        $\lesssim$              & \textbackslash lesssim            & $\gtrsim$         & \textbackslash gtrsim         &
        $\triangleq$            & \textbackslash triangle                                                               \\
        $\lessapprox$           & \textbackslash lessapprox         & $\gtrapprox$      & \textbackslash gtrapprox      &
        $\bumpeq$               & \textbackslash bumpeq                                                                 \\
        $\lessgtr$              & \textbackslash lessgtr            & $\gtrless$        & \textbackslash gtrless        &
        $\Bumpeq$               & \textbackslash Bumpeq                                                                 \\
        $\lesseqgtr$            & \textbackslash lesseqgtr          & $\gtreqless$      & \textbackslash gtreqless      &
        $\thicksim$             & \textbackslash thicksim                                                               \\
        $\lesseqqgtr$           & \textbackslash lesseqqgtr         & $\gtreqqless$     & \textbackslash gtreqqless     &
        $\thickapprox$          & \textbackslash thickapprox                                                            \\
        $\preccurlyeq$          & \textbackslash preccurlyeq        & $\succcurlyeq$    & \textbackslash succcurlyeq    &
        $\approxeq$             & \textbackslash approxeq                                                               \\
        $\curlyeqprec$          & \textbackslash curlyeqprec        & $\curlyeqsucc$    & \textbackslash curlyeqsucc    &
        $\backsim$              & \textbackslash backsim                                                                \\
        $\precsim$              & \textbackslash precsim            & $\succsim$        & \textbackslash succsim        &
        $\backsimeq$            & \textbackslash backsimeq                                                              \\
        $\precapprox$           & \textbackslash precapprox         & $\succapprox$     & \textbackslash succapprox     &
        $\vDash$                & \textbackslash vDash                                                                  \\
        $\subseteqq$            & \textbackslash subseteqq          & $\supseteqq$      & \textbackslash supseteqq      &
        $\Vdash$                & \textbackslash Vdash                                                                  \\
        $\shortparallel$        & \textbackslash shortparallel      & $\Supset$         & \textbackslash Supset         &
        $\Vvdash$               & \textbackslash Vvdash                                                                 \\
        $\blacktriangleleft$    & \textbackslash blacktriangleleft  & $\sqsupset$       & \textbackslash sqsupset       &
        $\backepsilon$          & \textbackslash backepsilon                                                            \\
        $\vartriangleright$     & \textbackslash vartriangleright   & $\because$        & \textbackslash because        &
        $\varpropto$            & \textbackslash varpropto                                                              \\
        $\blacktriangleright$   & \textbackslash blacktriangleright & $\Subset$         & \textbackslash Subset         &
        $\between$              & \textbackslash between                                                                \\
        $\trianglerighteq$      & \textbackslash trianglerighteq    & $\smallfrown$     & \textbackslash smallfrown     &
        $\pitchfork$            & \textbackslash pitchfork                                                              \\
        $\vartriangleleft$      & \textbackslash vartriangleleft    & $\shortmid$       & \textbackslash shortmid       &
        $\smallsmile$           & \textbackslash smallsmile                                                             \\
        $\trianglelefteq$       & \textbackslash trianglelefteq     & $\therefore$      & \textbackslash therefore      &
        $\sqsubset$             & \textbackslash sqsubset                                                               \\
        \bottomrule
        \end{tabular}
    \end{table}
    \flushleft
    \LaTeX 还提供了自定义二元关系符的命令 \textbackslash stackrel,用于将一个符号叠加在原有的二元关 系符之上:\par
    \[
        f_n(x) \stackrel{*}{\approx} 1    
    \]

    \newpage
    \appendix
    \section{附录}
\end{document}
