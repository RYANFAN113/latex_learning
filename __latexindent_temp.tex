%!TEX program = xelatex

%表示用xelatex编译文件
\documentclass[a4paper]{ctexart}
\usepackage{array}
\begin{document}
    \title{标题页}
    \author{Ryan\thanks{注脚}%
        \and Fan\thanks{注脚}%
        }
    \date{\today}
    \maketitle
    \abstract
    一般用于紧跟\textbackslash maketitle 命令之后介绍文档的摘要\par
    中文\LaTeX{}排版。
    \section{用\LaTeX 排版文字}
    {}分段\\换行\textbackslash\par
    冒号``please press the `x' key.''\par
    连字符-用来组成复合词,\\%
    短破折号--用来连接数字表示范围,\\%
    长破折号---用来连接单词\par
    省略号\dots{}和\ldots\par
    波浪号~\par
    强调\underline{文字,但是无法换行,}%
    \uline{ulem 宏包解决了这一问题,它提供的 uline 命令能够轻松生成自动换行的下划线。}%
    \emph{emph 命令用来将文字变为斜体以示强调。%
        \emph{如果在本身已经用 emph 命令强调的文字内部嵌套使用 emph 命令,}%
        内部则使用直立体文字。%
        }\par
    在合适的位置插入一个不会断行的空格Fig.~1, Ryan~Fan\par
    断行\\[15pt]可以带可选参数 ⟨length⟩,用于在换行处向下增加垂直间距%
    \newline{}或者newline命令,不用带参数\par
    \newpage 断页,在在双栏排版中只起到另起一栏的作用\par
    断词 I think this is: supercalifragil\-isticexpialidocious. %
    And I think this is: supercalifragilisticexpialidocious.\par
    \newpage
    \tableofcontents
    \section[目录和页眉页脚]{章节}
    \subsection{子章节}
    \subsubsection{子子章节}
    \paragraph{段落}
    \subparagraph{子段落}
    \section*{标题不带编号}
    \addcontentsline{toc}{section}{标题不带编号}
    \part{分块}
    \section{交叉引用}
    A reference to this subsection\label{sec:this} looks like: %
    ``see section~\ref{sec:this} on page~\pageref{sec:this}.''
    \section{脚注和边注}
    “天地玄黄,宇宙洪荒。日月盈昃,辰宿列张。”\footnote{出自《千字文》。}\par
    \begin{tabular}{l}
        \hline
        有些情况下(比如在表格环境、各种盒子\\
        内)使用 footnote 并不能正确生成脚\\
        注。我们以分两步进行,先使用 \\
        footnotemark 为脚注计数,再在合适\\
        的位置用 footnotetext 生成脚注。\footnotemark\\
        \hline
    \end{tabular}
    \footnotetext{出自《千字文》。}
    \marginpar{\footnotesize 边注较窄,不要写过多文字,最好设置较小的字号。}
    \section{特殊环境}
    \subsection{列表}
    有序列表
    \begin{enumerate}
        \item An item.
        \begin{enumerate}
            \item A nested item.\label{itref}
            \item[*] A starred item.
        \end{enumerate}
        \item Reference(\ref{itref}).
    \end{enumerate}
    无序列表
    \begin{itemize}
        \item An item.
        \begin{itemize}
            \item A nested item.
            \item[+] A `plus' item. + A ‘plus’ item.
            \item Another item. – Another item.
        \end{itemize}
        \item Go back to upper level.
    \end{itemize}
    关键字环境
    \begin{description}
        \item[Enumerate] Numbered list.
        \item[Itemize] Non-numbered list.
    \end{description}
    重定义无序列表的符号
    \renewcommand{\labelitemi}{\dag}
    \renewcommand{\labelitemii}{\ddag}
    \begin{itemize}
        \item First item
        \begin{itemize} 
            \item Subitem
            \item Subitem
        \end{itemize}
        \item Second item
    \end{itemize}
    重定义有序列表的符号
    \renewcommand{\labelenumi}{\Alph{enumi}>}
    \begin{enumerate}
        \item First item
        \item Second item
    \end{enumerate}
    \subsection{对齐环境}
    center、flushleft 和 flushright 环境分别用于生成%
    居中、左对齐和右对齐的文本环境。
    \begin{center}
        Centered text using a
        \verb|center| environment.
    \end{center}
    \begin{flushleft}
        Left-aligned text using a
        \verb|flushleft| environment.
    \end{flushleft}
    \begin{flushright}
        Right-aligned text using a
        \verb|flushright| environment.
    \end{flushright}
    还可以用以下命令直接改变文字的对齐方式:
    \centering
    Centered text paragraph.\par
    \raggedright
    Left-aligned text paragraph.\par
    \raggedleft
    Right-aligned text paragraph.\par
    \begin{flushleft}
        center 等环境会在上下文产生一个额外间距,%
        而 \textbackslash centering 等命令不产生,只是改变对齐方式。%
    \end{flushleft}
    \raggedright
    比如在浮动体环境 table 或 figure 内实现居中对齐,%
    用 \textbackslash centering 命令即可,没必要再用 center 环境。
    \subsection{引用环境}
    \begin{description}
        \item[quote] 用于引用较短的文字,首行不缩进\\
        Francis Bacon says:
        \begin{quote}
            Knowledge is power.
        \end{quote}
        \item[quotation] 用于引用若干段文字,首行缩进\\
        《木兰诗》:
        \begin{quotation}
            万里赴戎机,关山度若飞。
            朔气传金柝,寒光照铁衣。
            将军百战死,壮士十年归。

            归来见天子,天子坐明堂。
            策勋十二转,赏赐百千强。......
        \end{quotation} 
    \end{description}
    \subsection{代码环境}
    \begin{verbatim}
        #include <iostream>
        int main() 
        {
        std::cout << "Hello, world!"
                    << std::endl;
        return 0;
        }
    \end{verbatim}
    \begin{verbatim*}
        for (int i=0; i<4; ++i)
        printf("Number %d\n",i);
    \end{verbatim*}
    要排版简短的代码或关键字\textbackslash verb ⟨delim⟩⟨code⟩⟨delim⟩\par
    ⟨delim⟩ 标明代码的分界位置,前后必须一致,除字母、空格或星号外,%
    可任意选择使得不与代码本身冲突,习惯上使用 | 符号。\par
    \verb|\LaTeX| \\ 
    \verb+(a || b)+ \verb*+(a || b)+    
    \subsection{表格}
    \subsubsection{列表格}
    tabular 环境使用 ⟨column-spec⟩ 参数指定表格的列数以及每列的格式。\par
    \begin{tabular}{lcr|p{6em}}
        \hline
        left & center & right & par box with fixed width\\
        L    & C      & R     & P\\
        \hline
    \end{tabular}\par
    @ 格式可在单元格前后插入任意的文本,%
    但同时它也消除了单元格前后额外添加的间距。\par
    \begin{tabular}{@{} r@{:}lr @{}}
        \hline
        1 & 1 & one\\
        11 & 3 & eleven\\
        \hline
    \end{tabular}\par
    格式参数重复\par
    \begin{tabular}{|*{5}{c|}*{2}{p{3em}|}}
        \hline
        one & two & three & four & five & Hello! \LaTeX & Hello!\\
        1   & 2   & 3     & 4    & 5    & hello!        & \LaTeX\\
        \hline    
    \end{tabular}\par
    辅助格式 > 和 <,用于给列格式前后加上修饰命令\par
    \begin{tabular}{>{\itshape}r<{*}l}
        %需要使用array宏包
        \hline
        italic & normal \\
        column & column \\
        \hline    
    \end{tabular}\par
    \begin{tabular}{>{\centering\arraybackslash}p{16em}}
        \hline
        辅助格式甚至支持插入 \textbackslash centering 等%
        命令改变 p 列格式的对齐方式,一般还要加额外的命令 %
        \textbackslash arraybackslash 以免出错。\\
        \hline
        \textbackslash centering 等对齐命令会破坏表格环境里 %
        \textbackslash\textbackslash 换行命令的定义,%
        \textbackslash arraybackslash 用来恢复之。%
        如果不加 \textbackslash arraybackslash 命令,%
        也可以用 \textbackslash tabularnewline 命令%
        代替原来的 \textbackslash\textbackslash 实现表格换行。\\
        \hline
    \end{tabular}
    \LaTeX 本身提供了 tabular* 环境用来排版定宽表格,但是不太方便使用,%
    比如要用到 @ 格式插入额外命令,令单元格之间的间距为 \textbackslash fill,%
    但即使这样仍然有瑕疵:
    \begin{tabular*}{14em}{@{\extracolsep{\fill}}|c|c|c|c|}
        \hline
        A & B & C & D \\
        \hline
        a & b & c & d \\
        \hline
    \end{tabular*}
    is this… a test
    \appendix
    \section{附录}
\end{document}
